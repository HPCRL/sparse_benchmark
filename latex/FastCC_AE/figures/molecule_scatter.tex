\begin{subfigure}{\columnwidth}
\pgfplotstableread[col sep=comma,]{data/molecule_scatter.txt}\datatable
\begin{tikzpicture}
\begin{axis}[
    moleculeScatterPlot,
    %xticklabels from table={\datatable}{Bytes},
    xmode=log,
    ymode=log,
    legend to name=moleculeScatterLegend,
    legend columns = 3,
    legend style={font=\small},
    ]



    %Caffeine-vvoo
    \addplot[NIPS013] table [x=tile, y=caffeine-vvoo]{\datatable};
    \addlegendentry{c-vvoo}

    %Caffeine-vvov
    \addplot[Chicago0] table [x=tile, y=caffeine-vvov]{\datatable};
    \addlegendentry{Caffeine-vvov}

    %Caffeine-ovov
    \addplot[Chicago01] table [x=tile, y=caffeine-ovov]{\datatable};
    \addlegendentry{Caffeine-ovov}

    %Guanine-vvoo
    \addplot[Chicago123] table [x=tile, y=guanine-vvoo]{\datatable};
    \addlegendentry{Guanine-vvoo}

    %Guanine-vvov
    \addplot[Vast01] table [x=tile, y=guanine-vvov]{\datatable};
    \addlegendentry{Guanine-vvov}

    %Guanine-ovov
    \addplot[Vast014] table [x=tile, y=guanine-ovov]{\datatable};
    \addlegendentry{Guanine-ovov}



\end{axis}
\end{tikzpicture}


\caption{Execution time variation with tile size: quantum chemistry}
\label{fig:frostt_u}
\label{fig:dlpno_u}
\label{fig:molecule_scatter_subfig}

\end{subfigure}